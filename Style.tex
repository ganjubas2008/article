\documentclass{article}


%INCLUDE
\usepackage[a4paper, left=20mm, right=20mm, top=20mm, bottom=20mm]{geometry}

\usepackage{amsmath, amsthm}
\usepackage[T1,T2A]{fontenc}
\usepackage[utf8]{inputenc}
%\usepackage[russian]{babel}
\usepackage{amssymb}
\usepackage{hyperref}
\usepackage{multirow}
\usepackage{stackengine}
\usepackage{algorithm}
\usepackage{algpseudocode}
\usepackage{lipsum}
\usepackage{authblk}
\usepackage{graphicx}
\usepackage{multicol}
\usepackage{pgfplots}
\usepackage{bm}
\usepackage{subcaption}
\usepackage{float} % for the H specifier
\usepackage{amsmath} % For \mathscr
\usepackage{mathrsfs} % For \mathscr
\usepackage{calrsfs} % For \mathcal
\usepackage{dutchcal} % For \mathdutchcal
\usepackage{nicematrix}
\usepackage{changepage}
\usepackage{threeparttable}



%KHALED BEGIN

\usepackage{natbib}
\setcitestyle{authoryear,round,citesep={;},aysep={,},yysep={;}}

\renewcommand{\bibname}{References}
\renewcommand{\bibsection}{\subsubsection*{\bibname}}
\bibliographystyle{plainnat}

\usepackage[tableposition=top]{caption}
\usepackage{amsmath,amsthm,amssymb,mathtools,graphicx,enumitem,hyphenat,float, threeparttable}
\usepackage{mathabx, hyperref}
\hypersetup{ %
    pdfborder=0 0 0,
    pdfpagemode=UseNone,
    colorlinks=true,
    linkcolor=blue,
    citecolor=blue,
    filecolor=blue,
    urlcolor=blue,
    pdfview=FitH}
%KHALED END

%THEOREMS
\makeatletter
\def\th@plain{%
  \thm@notefont{}% same as heading font
}
\makeatother

\theoremstyle{plain}
\newtheorem{lemma}{Lemma}
\newtheorem{sublemma}{Sublemma}[lemma]


\theoremstyle{plain}
\newtheorem{assumption}{Assumption}

\theoremstyle{plain}
\newtheorem{statement}{Statement}

\theoremstyle{plain}
\newtheorem{corollary}{Corollary}

\theoremstyle{remark}
\newtheorem*{remark}{Remark}

\theoremstyle{plain}
\newtheorem{theorem}{Theorem}


%COMMANDS
\newcommand{\norm}[1]{\left\|#1\right\|}
\newcommand{\inner}[2]{\langle #1, #2\rangle}
\newcommand{\summ}[3]{\sum^{ #2 }_{ #1 } #3}
\newcommand{\fsumm}[3]{\frac{1}{M}\sum^{ #2 }_{ #1 } #3}
\newcommand{\coef}[1]{\biggr[ #1 \biggr]}
\newcommand{\E}{\mathbb{E}}

\newcommand{\mc}{\mathbcal}

\definecolor{darkgreen}{RGB}{0, 128, 0} % Define a darker shade of green
\newcommand{\greencheck}{{\color{darkgreen}\checkmark}} % Define the command with the darker green color
%\newcommand{\greencheck}{{\color{green}\checkmark}}
\newcommand{\redcross}{{\color{red}\textbf{\texttimes}}}
\newcommand{\smallfont}[1]{{\small #1}}

\def\O{\mathcal{O}}
\DeclareMathOperator*{\argmin}{arg\,min}